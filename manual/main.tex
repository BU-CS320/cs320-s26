\documentclass[10pt]{report}
\usepackage{manual}

\title{
  {\huge Course Manual} \\
  \vspace{2cm}
  {\large \texttt{CAS CS 320}: \textit{Concepts of Programming Languages}} \\
  {\large Boston University}
}
\date{Spring 2026}

\begin{document}

\maketitle
\tableofcontents

\abstract{

  This is a manual for the course \texttt{CAS CS 320}:
  \textit{Concepts of Programming Languages}.  It contains a general
  overview of the course and its policies.  It does \textit{not}
  contain specifics about the material being covered in the course;
  this appears on the
  \href{https://bu-cs320.github.io/cs320-s26/index.html}{course
    webpage}.  \Cref{fig:overview} contains an overview of the course.

  \begin{figure}
    \begin{tabular}{|l|l|}
      \hline
      Course Code & \texttt{CAS CS 320} \\
      Course Title & Concepts of Programming Languages \\
      Semester & Spring 2026 \\
      Instructor & Nathan Mull \\
      Teaching Fellow & Anthony DeRossi \\
      Teaching Assistant & Vivian Tvedt \\
      Course Assistants & Miranda Quimbar and Shiyun Yang \\
      Meeting Times
      & Tuesday and Thursday, 2:00PM-3:15PM (A1) \\
      & Tuesday and Thursday, 3:30PM-4:45PM (A2) \\
      Meeting Location
      & \href{https://www.bu.edu/classrooms/classroom/sci-113/}{SCI 113} (A1) \\
      & \href{https://www.bu.edu/classrooms/classroom/sar-103/}{SAR 103} (A2) \\
      Midterm Dates
      & February 19 (during lecture) \\
      & March 31 (during lecture) \\
      Grade Breakdown
      & 05\% Workshops \\
      & 05\% Labs \\
      & 20\% Assignments \\
      & 20\% Mini-projects \\
      & 30\% Midterm Exams \\
      & 20\% Final Exam \\
      \hline
    \end{tabular}
    \centering
    \caption{Course overview}
    \label{fig:overview}
  \end{figure}

}

\chapter{Week 0 To-Do list}

You should complete the following items within the first 48 hours of
the start of the semester.  Please reach out if you have concerns
about any of the items listed.

\begin{itemize}
\item[$\square$] Verify that you have access to a laptop computer
  during the semester
\item[$\square$] Verify that you know where the lecture is held
\item[$\square$] Verify that you know where the discussion section in
  which you're registered is held
\item[$\square$] Join Piazza with the following
  \href{https://piazza.com/bu/spring2026/cascs320}{sign-up link}
\item[$\square$] Join Gradescope with the following
  \href{https://www.gradescope.com/courses/1222812}{sign-up link}
  (Entry code: V44GKB)
\item[$\square$] Familiarize yourself with the
  \href{https://bu-cs320.github.io/cs320-s26/index.html}{course
    webpage}
\item[$\square$] \textit{(Optional)} Add the Piazza, Gradescope, and
  course webpage as bookmarks in your Internet browser
\item[$\square$] Review the course calendar and determine which office
  hours you're able to attend
\item[$\square$] \textit{(Optional)} Add the course calendar to your
  own calendar
\item[$\square$] Read this manual in its entirety
\item[$\square$] Submit the assignment on Gradescope confirming that
  you've read this manual
\item[$\square$] If necessary, review material from \texttt{CAS CS
  210}: \textit{Computer Systems} on how to work in the terminal
\end{itemize}

\chapter{General Information}

\texttt{CAS CS 320}: \textit{Concepts of Programming Languages} is a
course \textit{about} programming languages, particularly the
\textbf{design} and \textbf{implementation} of programming
languages. In this course, we take up the programming language as an
object of formal study. This course is \textit{not} about how to program,
though the principles we cover are generally useful for writing and
reasoning about programs.

The first part of the course is on functional programming in OCaml,
based on \texttt{CS3110}: \textit{Data Structures and Functional
  Programming} at Cornell University. The topics include functional
design, algebraic data types, higher-order functions, polymorphism,
functional data structures, and modular programming. It’s during this
part that we learn to \textit{think} functionally, to view programs
not as sequences of commands manipulating global state, but as
compositions of functions that deconstruct and reorient data.  We also
introduce in this part of the course fundamental concepts from the
theory of programming languages\textemdash like inference systems,
typing rules, semantic rules, and derivations\textemdash with an eye
towards how these concepts apply to OCaml in particular.

In the second part of the course we implement several interpreters for
fragments of OCaml; that is, we write OCaml programs that
\textit{execute} other OCaml programs. The topics covered include
formal grammar, parsing, operational semantics, variable scope and
binding, closures, type checking, type safety, and type inference. By
the end of the course, you’ll be able to execute some of the simpler
programs we wrote in the first part of the course using your own
interpreter.  You'll also be able to reason formally about what a
programming language is, and what makes some programming languages
good and others not-so-much.

\section{Prerequisites}

The formal prerequisites for this course are:
\begin{itemize}
\item \texttt{CAS CS 111} \& \texttt{112}: \textit{Introduction to Computer Science}
\item \texttt{CAS CS 131}: \textit{Combinatoric Structures}
\item \texttt{CAS CS 210}: \textit{Computer Systems}
\end{itemize}
You’ll get the most out of this course (and you'll have the best time
of it) if you’ve completed the 200 level major requirements for the
computer science degree at BU. Experience with a high-level
programming language (like Python or Java) is a must.  The dependence
on \texttt{CS131} is a level of mathematical maturity and
exposure to the principle of induction.  We don't depend explicitly on
the material of \texttt{CS210}, but it will be useful in
appreciating the second half of the course.  We also assume general
comfort in the terminal.  If you don't have experience with this, then
you'll be have to self-learn it within the first week of the course.

\section{Learning Objectives}

From this course we hope that you will:
\begin{itemize}
\item
  Learn the rudiments of OCaml, a functional programming language
  which is likely very different from other languages you’ve learned.
  In particular, we'd like you learn to \enquote{think functionally}
  and understand the benefits of doing so.
\item
  Learn to read and write formal specifications for programming
  languages.
\item
  Gain an understanding of the theoretical basis for what makes a
  \enquote{good} programming language.
\item
  Gain an appreciation for what goes into the development of the
  programming languages we use everyday, particularly by implementing
  interpreters for a collection of OCaml-like languages.
\end{itemize}
\texttt{CS320} fulfills a single unit in the BU Hub area
\textbf{Creativity/Innovation}.  This deserves some
explanation. Programming language design is a (perhaps surprisingly)
creative endeavor, in which personal aesthetics play a large role in
what features are included (or not) in a language.  A fair amount of
creativity also goes into discovering new features that aid the
safety, productivity, or ergonomics of a new
language.\footnote{\href{https://rust-lang.org}{Rust} is a great
example of a modern language with a fair amount of creativity behind
its design.} This is in part to say that what we cover is based on our
notion of what a \enquote{good} programming language is, which won’t
necessarily align with your own views or the views of other
programming language designers. This also must be understood in spite
of the fact that much of what we do in the course will not feel very
creative. All I can say on this point: you have to know what came
first before you can change the state-of-the-art.

One more point I'd like to make. It's no secret that \texttt{CS320} is
not everyone's first choice for a course to take in our
department. OCaml is not a terribly popular language, and programming
language design doesn't have the same hype as machine learning or data
science.  Now, I wouldn't go as far as saying that OCaml and
programming language design are \enquote{on the rise,} but there are
some interesting trends worth noting.

\begin{itemize}
\item OCaml is being heavily used in and developed by the trading firm
  \href{https://www.janestreet.com}{Jane Street}, and is being used to
  a lesser extent by companies like Bloomberg, Docker, and Facebook,
  among others.
\item
  OCaml won the ACM SIGPLAN's 2023 Programming Languages Software
  Award.\footnote{\href{https://ocaml.org/news/sigplan_announcement}{\texttt{https://ocaml.org/news/sigplan\_announcement}}}
\item
  In LinkedIn's ranking of the 50 top colleges in the US for long-term
  career success, one of the \enquote{most notable skills} offered by
  BU (university-wide) is
  OCaml.\footnote{\href{https://www.linkedin.com/pulse/linkedin-top-colleges-2025-50-best-long-term-career-success-kritf/}{\texttt{https://www.linkedin.com/pulse/linkedin-top-colleges-2025-50-best-long-term-career-success-kritf/}}}
\item
  I've been told that OCaml may become a language which is used in
  competitive programming.
\item
  Rust (a fairly new systems programming language with a lot of hype)
  was originally written in OCaml and borrows many of its functional
  features. In the other direction, Jane Street recently came out with
  \href{https://oxcaml.org}{OxCaml}, an extension to OCaml with
  Rust-like features.
\item Rust itself is an fascinating contemporary example of
  programming language design having a major impact on the world of
  software engineering.
\item
  Domain specific language and compilers are becoming more common
  tools for dealing with bottlenecks in high-level languages, in areas
  as wide-ranging as machine learning, networks, and scientific
  computing.
\end{itemize}
Just a couple things to think about as you're determining what skills
are worth having in your toolbox.

\section{Course Structure}

\subsection*{Lectures}

We hold lectures each week on Tuesday and Thursdays (see the registrar
and the course webpage for details).  During lecture, we cover
the material that is presented in the reading, do live coding
examples, and provide practice problems.  The material used in the
lecture is made available on the course webpage before the lecture
meeting.  Barring technical difficulties, recordings of the lecture
will be made available.

We won't take attendance during lecture (except during
Workshops\textemdash see below for further details) but it is highly
recommended that you attend, and refer you to the BU
\href{https://www.bu.edu/academics/policies/attendance/}{Attendance}
policy.  You'll be expected to participate in lectures by working on
practice problems and occasionally discussing topics with the people
sitting around you.

\subsection{Workshops}

Six lectures during the semesters will be run as workshops. This means
that you will be expected to do some at-home learning beforehand; we
will spend a majority of the lecture time working on an in-class
activity in groups. We take attendance insofar as you will have to
submit the required material by the end of lecture.

\subsection*{Discussion Sections}

We hold discussion sections each week on Monday (see the registrar and
the course webpage for details).  We take attendance insofar as you
will have to submit the required material by the end of lab.

\section{Resources}

\subsection*{Material}

The first half of the course uses the textbook
\href{https://cs3110.github.io/textbook/cover.html}{OCaml Programming:
  Correct + Efficient + Beautiful}. All other course material will be
made available on the course
\href{https://github.com/BU-CS320/cs320-fall-2025}{GitHub repository}
and on the course webpage. If you are unfamiliar with git and GitHub,
see the \href{https://docs.github.com/en/get-started}{GitHub
  documentation} for information and tutorials. Please check the
course webpage and repository frequently.  We'll also be testing out
some new notes for this course. These will be made available on the
course webpage.

\subsection*{Programming}

The programming in this course is done in OCaml.  You're required to
set this up on you personal machine or on a machine that you'll have
access to throughout the semester.  You'll have an opportunity to get
help with this during your first discussion section.  Please attend
office hours and use Piazza if you need help troubleshooting. If
you're are worried about access to technology, please contact me as
soon as possible and we can see what we can do (though I cannot make
any guarantees).

\subsection*{Course Communication}

Course announcements and discussions will happen on Piazza. If you're
unfamiliar with Piazza, see their
\href{https://support.piazza.com/support/solutions/48000185443}{support
  page} for information and tutorials. Some policies regarding the use
of Piazza:
\begin{itemize}
\item
  \textit{Don't ask homework questions directly.} Formulate a question which
  will aid in your understanding, and will hopefully help others as
  well.
\item
  \textit{Don't give homework solutions directly.}
\item
  \textit{Piazza is as useful as it is active.} Teaching fellows and course
  assistants will be answering questions on Piazza, but don't hesitate
  to answer questions yourself.
\end{itemize}
Make sure to set notifications correctly so you can keep up with
updates regarding the course. \enquote{I didn't see the Piazza post
  about it} is never a valid excuse for missing a piece of
information.

\subsection*{Submission}

We'll be using Gradescope for assignment submissions. If you are
unfamiliar with Gradescope, see their
\href{https://www.gradescope.com/get_started}{Get Started} page for
information and tutorials.

\chapter{Evaluation}

\begin{figure}
  \begin{tabular}{|l|l|}
    \hline
    05\% & Workshops (6 total, 1 dropped) \\
    05\% & Labs (13 total, 3 dropped) \\
    20\% & Assignments (6 total, 1 dropped) \\
    20\% & Mini-projects (3 total) \\
    30\% & Midterm Exams \\
    20\% & Final Exam \\
    \hline
  \end{tabular}
  \centering
  \caption{Grade breakdown}
  \label{fig:grades}
\end{figure}

The grading breakdown for this course is given in \Cref{fig:grades}.
The sites of evaluation are detailed in the following sections.  Your
raw percentage will be determined according to this breakdown and your
final letter grade is guaranteed to be at least what is determined by
Wheelock College's
\href{https://www.bu.edu/academics/wheelock/policies/grades-course-credits-incomplete-coursework/}{Grading
  Scale}.\footnote{Formally we're part of the College of Arts and
Sciences (CAS), but this grading scale is standard.}  But, to borrow a
phrase from Professor Mark Bun: \enquote{to correct for the
  possibility of [quizzes] and exams being more difficult than
  anticipated, letter grades may be (significantly) increased above
  these guarantees.}  Specifically, we may retroactively curve exam
and quiz grades using a linear scale.\footnote{See
\href{https://divisbyzero.com/2008/12/22/how-to-curve-an-exam-and-assign-grades/}{this
  article} for details if you're interested.}

\section{Workshops}

Each workshop will have an in-class assignment that you will be
required to submit at the end of the lecture period.  There are 6
workshops total.  We drop your lowest workshop assignment grade, so
only 5 workshop assignments count towards your final grade.

\section{Labs}

Each lab will have an in-class assignment that you will be required to
submit at the end of the discussion section.  There are 13 labs total.
We drop your lowest 3 lab grades, so only 10 workshop assignments
count towards your final grade.

\section{Assignments}

Assignments are released weekly on Thursdays during the first half of
the semester and are due a week later on the following Thursday by
8:00PM.  See the calendar on the course webpage for details.
Assignments consist of written problems that are to be submitted as a
pdf file via Gradescope, as well as programming problems that will be
autograded.  There are 6 assignments total.  We drop your lowest
assignment grade, so only 5 assignments count towards your final grade
in the course.  We don't accept late assignments under any
circumstances.

You'll notice that, despite the fact that there are many assignments,
they account for a very small portion of your final grade.  You should
think of the assignments in this course as \textit{accountability
  checks} in that they require to engage with the material each week.
But the more effort you put into learning and internalizing the
material in the assignments, the more successful you'll be in the
other evaluation sites of the course, like the quizzes and
exams.\footnote{We understand that this approach to evaluation will
not be universally liked, and that in-person evaluation can be
challenging.  We take this under careful consideration when we
calibrate the difficulty of the material and the workload we expect.}

\section{Mini-Projects}

Mini-projects are released (roughly) every other week on Thursdays in
the second half of the course, and are due two weeks later on Thursday
by 8:00PM.  See the calendar on the course webpage for details. Each
mini-project requires you to build an interpreter. Each interpreter
will be slightly more complex than the last. You should think of these
as roughly the same amount of work as two assignments, but with a
single large programming task (building an interpreter) instead of
small programming tasks.  Mini-projects will be accompanied with a
check-in, which is another form of \textit{accountability check} to
verify you're making progress on the project.  There are 3
mini-projects total.  \textbf{It's not possible to drop a
  mini-project.}

\section{Midterm Exams}

The midterm exams will be held on February 19 and March 31 during
lecture.  They will be a closed-note written exams.  The grades of the
midterm exams will be weighted as follows:

\begin{displaymath}
  20\% \times \max (s_1, s_2) + 10\% \times \min (s_1, s_2)
\end{displaymath}
where $s_1$ is the (normalized) score recieved on the first midterm
and $s_2$ is the (normalized) score recieved on the second.  In other
words, we offer a small amount of grade forgiveness by weighting your
better performance more favorably.

\section{Final Exam}

The date of the final exam will be determined later in the semester.
It will be a closed-note written exam.  It is meant to verify that you've
internalized the basic concepts of the course, and can also apply them
to solve novel problems.

\chapter{Policies}

There are a number of policies associated with this course, some
specific to the course and others which hold more generally in the
university.  These policies are detailed below in the following
sections.


\section{Diversity Statement}

Our aim is to present material in a way that respects the diversity of
the student body. If we fail to do this, please make us aware. Any
suggestions are welcome and appreciated. We also expect students to
appreciate and respect the unique opportunity they have to participate
in a diverse student body like ours.

\section{Disability Statement}

If you require disability accommodations, please contact us as soon as
possible. You should provide us with the appropriate documentation,
available through the \href{https://www.bu.edu/disability/}{Disability
  and Access Services}.  In order to receive accommodations, you
\textit{must} be in contact with us.

If there’s a policy that we're failing to comply with, please
reach out with suggestions. And if you’d like accommodations that
aren't covered by existing services or policies, feel free to contact
us and we can see what we can do.  We want everyone to feel able to
fully participate in the course.

\section{Sexual Misconduct}

Please read the
\href{https://www.bu.edu/policies/sexual-misconduct-title-ix-hr/}{Sexual
  Misconduct Policy} and review the entire page for information on
talking to someone confidentially about experiences of sexual
misconduct, filing a report, and any other relevant information. Above
all, you should feel safe, and able to be productive. If this is not
the case, please reach out to us or someone else immediately.

The members of the course staff are considered \enquote{mandated
  reporters} and are required to report cases of sexual misconduct.
Therefore, \textbf{we cannot guarantee the confidentiality of a
report}. We must provide our Title IX coordinator with relevant details
such as the names of those involved in the incident.  The university
will consider a request for confidentiality and respect it to the
extent possible.

With that in mind, if you come to any of us with questions or
concerns, we will handle the situation to the best of our ability and
connect you with available resources.


\section{Academic Integrity}

Please read the
\href{https://www.bu.edu/academics/policies/academic-conduct-code/}{Academic
  Conduct Code} and review the entire page for information about what
constitutes academic dishonesty and what penalties arise as a result
of violations of this code.  This is taken very seriously at BU and we
take it seriously in this courses.  There are a couple policies about
which we'll be strict:
\begin{itemize}
\item
  You must submit your own work for all assignments and mini-projects.
  Submitting the same file as another student, or something notably
  similar (e.g., identical wording or code in large parts of the
  solution) is considered academic misconduct and will be handled
  accordingly.
\item
  Copying or information sharing regarding in-class evaluations like
  quizzes and exams is considered academic misconduct and will be
  handled accordingly.
\end{itemize}
If you work with others, consult materials found on the Internet, or
use an AI assistant, you should cite your sources. This is a useful
skill in any setting, and so we recommend being as conservative as
possible regarding citations. In any assignment, these citations
should appear next to every corresponding problem (in comments if the
submission is code). Some examples:

\begin{itemize}
\item
  I discussed problem 1 and 2 with Leah Smith. She helped me
  understand X and Y aspects of the problem.
\item
  I saw the stack overflow post
  stackoverflow.com/questions/6681284/python-numpy-arrays which
  informed my solution.
\item
  I helped Zihan Guo with problem 4. I told them to try using X.
\item
  I asked ChatGPT “what’s the largest eigenvalue of this
matrix?”
\end{itemize}
When in doubt, err on the side of longer, more descriptive citations.
We do not consider missing or poor citations is a direct act of
academic misconduct, but we will consider this grounds for further
investigation in suspicious cases.  Above all, use your best judgment
and remember:
\begin{itemize}
\item
  We care about your success in this course. We provide a number of
  avenues to ask for help, please use them.
\item
  You will have to answer questions on quizzes and exams without
  external aids (and in interviews when you apply for a job).
\item
  If you don’t know how to start thinking about a problem, it’s okay
  to ask for pointers in office hours and on Piazza.
\item
  We have safeguards (like dropped homework assignments) in the case
  you are unable to complete an assignment. In other words, don't
  submit someone else's work when you can drop an assignment.
\end{itemize}

\section{Generative AI}

The problem of generative AI in higher education will likely occupy us
for the next decade or so.  The role of these tools our lives is still
an open question, one with many possible answers.  But these tools
exists, and the university, for better or for worse, has made them
more accessible with the introduction of
\href{https://terriergpt.bu.edu/login}{TerrierGPT}, to which all
students of the university have access.  As such, all courses
(including ours) are reviewing their policies.  Keep in mind that this
is all one big experiment.  We don't know if our policy makes sense in
the long term (or even now).  But it's our attempt to come to terms
with the appearance of these tools in our courses.

The policy this semester: \textbf{The use of generative AI to produce
  solutions to assignments and mini-project is prohibited.} You are
allowed to use these tools to aid your learning.  We understand that
this is difficult to enforce, much of this will be on the honor
systems.  In an lower-division course, we might have a different
policy, but a this point in your academic career, we expect that you
are able to make the correct judgments as to the use of generative AI.
Also, as you might have noted, we've re-weighted evaluation sites in
order to put a greater emphasis on in-class closed-book evaluation.

An obligatory concluding remark: we know that existing models can
solve many of the problems we will ask you.  This does not negate the
value in knowing how to do them without the help of external tools.
To draw an imperfect analogy, we don't learn a new language in order
to have memorized a vast collection of words and grammar rules, but in
order to \textit{internalize} the language, and learn how to
\textit{interact} with it and in it.  This is our goal in this course
and beyond.  These tools can be incredibly useful in the process of
learning and internalizing.  But the internalizing is what we really
want to achieve, so we can see a problem and can \textit{sense} the
underlying structure to which we can apply our knowledge from this
course.

\section{Additional Attendance Policies}

As we've noted, we won't take attendance in our course. Instead, we
remind you that, according to the
\href{https://www.bu.edu/academics/policies/attendance/}{Attendance}
policy at BU, you're required to attend the courses in which you're
registered.

\subsection*{Absence Due to Religious Observance}

According to the BU policy on
\href{https://www.bu.edu/academics/policies/absence-for-religious-reasons/}{Absence
  Due to Religious Observance}: you \enquote{shall be excused from any
  such examination or study or work requirement, and shall be provided
  with an opportunity to make up such examination, study, or work
  requirement that may have been missed because of such absence on any
  particular day; provided, however, that such makeup examination or
  work shall not create an unreasonable burden upon such school.}

\subsection*{Bereavement}

According to the BU policy on
\href{https://www.bu.edu/academics/policies/student-bereavement/}{Student
  Bereavement}: you \enquote{should be granted up to five weekdays of
  bereavement leave for the death of an immediate family member.} Your advisor should help you coordinate your leave.

\section{Additional Grading Policies}

\subsection*{Regrade Requests}

Regrade requests may be submitted on Gradescope for up to one week
after receiving the grade for an evaluation site.
Regrade requests will only be considered in the case that the grader
has made a mistake in grading.  Any regrade requests which solely
appeal for a higher grade will not be considered.

\subsection*{Grading Grievances}

According to the BU policy on
\href{https://www.bu.edu/academics/policies/policy-on-grade-grievances-for-undergraduate-students-in-boston-university-courses}{Grade
  Grievances}: you may \enquote{contest a final course grade received
  in a unit-bearing Boston University course when that grade is
  alleged by the student to be arbitrary.}  Read the policy for more
information.  We recommend contacting us before submitting a formal
appeal.

\subsection*{Incomplete Grades}

According to the BU policy on \href{
 https://www.bu.edu/academics/policies/incomplete-coursework/}{Incomplete
  Coursework}: \enquote{An incomplete grade (I) is used only when the
  student has conferred with the instructor prior to the submission of
  grades and offered acceptable reasons for the incomplete work. An
  incomplete grade may be appropriate when the student has
  participated in and completed requirements representing a majority
  of the course, and circumstances prevent the student from completing
  remaining requirements by the conclusion of the course.} In
particular, \textbf{you must contact us before the last day of the
  semester in order to receive an incomplete grade.}

\chapter{Closing Remarks}

Quite a bit goes into organizing a course as well as taking a course
. In light of my comments on citations, I'll note that much of what's
in this document is based on similar documents (often taken without
permission) by Mark Crovella, Mark Bun, Jonathan Appavoo, Preethi
Narayanan, Ravi Chugh, Andrew McNutt, and others I may be missing.
All told, we hope that most of this logistical information will be
overshadowed in your memory by the concepts of the course, and that we
can focus on having a good time doing math and programming.

\section{Course Agreement}

In addition to a manual, we also consider this document a contract. The following
is what you must agree to in order to remain in this course.

\begin{quote}
  \textit{By enrolling in this course, I am agreeing to the policies
    outlined in this document, and I will uphold them to the best of
    my ability. I will also, generally speaking, try to be a
    reasonable person and be nice and good and respectful to the
    people around me taking\textemdash and running \textemdash the
    course.  In return, I expect a high-quality learning experience
    and respect from those around me taking\textemdash and
    running\textemdash the course.}
\end{quote}

\section{University Resources}

There are quite a few BU resources, it can sometimes be overwhelming.
Here’s a small list of the ones we think are important.  If you’re
struggling in this course due to personal/health conditions, we can’t
guarantee we can help, but if you’re comfortable reaching out, feel
free to send us an email and we can see if we can point you towards
the correct resources. If you’re not comfortable reaching out to us,
that’s okay too, hopefully this list can help you find what you need.
Also, keep in mind you can post anonymously on Piazza if you want to
ask for help without including your name.
\begin{itemize}
\item \href{https://www.bu.edu/disability/}{Disability and Access Services}
\item \href{https://www.bu.edu/shs/}{Student Health Services}
\item \href{https://www.bu.edu/shs/outreach-prevention/}{Outreach and Prevention}
\item \href{https://www.bu.edu/shs/mental-health/}{Behavioral Medicine}
\item \href{https://www.bu.edu/shs/survivor-support/}{Survivor Support (SARP)}
\item \href{https://www.bu.edu/advising/educational-resource-center/}{Educational Resources Center}
\item \href{https://www.bu.edu/isso/}{International Students \& Scholars Office}
\end{itemize}
\end{document}
